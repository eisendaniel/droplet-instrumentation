\chapter{Evaluation}\label{C:eval}


\section{Mechanical Stability}

\subsection{Procedure}
\begin{itemize}
    \item
    \item
    \item
\end{itemize}

\subsection{Results}

\section{Stepper Motor Driving}

\section{Droplet Volume}

\section{Repeatability and Reliability}

This section aims to produce the main set of comparable results for evaluating the projects produced system against the output of the previous setup. Thus justifying the success of one of main goals.

\subsection{Procedure}
\begin{itemize}
    \item \textbf{Initial Setup:} Roughly Position substrate stage, reservoir platform and note angular positions as well as vertical clearance requirements.
    \item \textbf{Zero System:} Using overhead camera precisely position pipette tip above substrate centre
    \item \textbf{Data Acquisition:} Initialise cameras, collect pixel:mm calibration data for analysis, initial LabView temperature logger, and environmental monitor noted. Info: temperature data rate, camera frame rate
    \item \textbf{Automated Sequence:} Via the serial link, enter the procedures command sequence to represent $\rightarrow$ lower, draw up fluid, raise, position over substrate, dispense, lower, raise, clear camera view. With appropriate delays.
    \item \textbf{Capture:} Begin data collection and automated dispense.
    \item \textbf{Repeat: } Minimum of 5 times. Each time carefully cleaning substrate surface to minimised up measured factors. 
\end{itemize}