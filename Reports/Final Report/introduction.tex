\chapter{Introduction}\label{C:intro}

\section{Motivation}
The investigation of droplet impact and evaporation is an area of interest and application to various industries. Examples of these include milk powder spray drying, inkjet printing, and applications of evaporative cooling. This project will continue on from a previous instrumentation setup, evaluating its shortcomings, and designing the next generation. This is done with the intention to improve the reliability and usability of the collected data from the previous setup by better controlling for identified affecters, and introduce methods of automating the process.

\section{Proposed Approach}
This project proposes to build the next generation of this droplet instrumentation system, based on automation and motorisation with the express aim to improve repeatability, usability and reliability by controlling for variables that affects droplet position and morphology. More precisely the new system will use stepper motors and a serial control interface to automate the positioning of a droplet dispensing electronic pipette and electronically interface with it to enable remote droplet dispensing.     

\section{Goals}
The goals of the experiment are to characterise the behaviour of a droplet impacting and evaporating from a given substrate. This forms the backbone for the goals of this project:
\begin{itemize}
    \item Increased Repeatability and Reliability droplet position, volume and morphology with the intent that controlling these variables results in more consistent results. 
    \item Increase speed and usability of the experimental process via automation to enable more data to be collected more easily and consistently. 
    \item Produce an expandable platform that can be built upon beyond the scope of this project.
\end{itemize}

\section{Evaluation}
Evaluation of the success of this project is split in two:

Firstly, individual system components will be unit tested against their specifications and through the lens of improving system repeatability. I.e. Has the mechanical been designed to minimise vibration and resonance.

Secondly, as this project exists within the greater context of an existing experimental process, data from the initial system is used to form a baseline for a comparative evaluation of the project integrated system.    