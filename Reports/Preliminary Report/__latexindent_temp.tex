\chapter{Background}\label{C:back}

\section{Current Experimental Setup}
The instrumentation as exists is assembled with an optical breadboard and XYZ(R) stages with micrometre controls. A central stage holds the substrate with internally mounted thermocouple. 2 manual focus cameras are positioned in profile and top down, and the droplet is dispensed manually via a syringe mounted horizontally to a XYZ+R stage.
The current procedure is a manual process. The syringe tip is rotated above a marked point on the substrate, and hand empties and refilled. This results in volume and positional variation run to run. These manual procedural variance is the focus of this project.

\begin{table}[h]
    \centering
    \begin{tabular}{|l|l|l|}
    \hline
                   & \textit{distance}          & \textit{measure temperate drop} \\ \hline
    Min Pos Offset & 0.71mm                     &           -1.790142
    \\ \hline
    Max Pos Offset &                            &                                 \\ \hline
                   & \textit{Volume}            & \textit{Contact Angle}          \\ \hline
    Shape Varience & $0.629mm^3$ & 25.894 degrees *                \\ \hline
    \end{tabular}
    \caption{Variance in Setup}
    \subcaption*{* note: This excluded an outlier of a almost sphereical droplet with contact angles exceeding 95 degrees}
    \end{table}

Date taken from a series of 5 droplet runs was analyses to extract the

\section{Existing Solutions}
Brief discussion of other similar experiments and rig and what problems they solve with what methods

\newpage
\section{Background on Stepper motor control}
This section will cover the background of controlling bipolar stepper motors via a step/direction style driver setup as its concepts will be mentioned later in the report. This is a focused background on the key considerations and requirements when designing for and operating this specific subset and by no mean applicable to all driving and specific motor choices.

The stepper motors are provide precise positioning and are capable of moving their rotor to a specified position and holding that position irrespective of the load torque. This capability makes the stepper motors to be used in optics, medical instruments, factory automation, and industrial equipment. The typical topology of a stepper driving system (based around the step/direction method) consists of a controller, driver, and stepper motor. The controller provides a direction signal and step pulses and direction signal, while the driver converts these signals into actual electrical power and supplies them to the motor. The stepper motor moves in steps, each step covering one step angle, which can be described as the rotor displacement corresponding to one step pulse \cite{step_app}.

Stepper motors typically have a step size specification (e.g. 1.8° or 200 steps per revolution), which applies to full steps. Step/Direction driver usually provide a 'microstepping' mode increases this resolutions by allowing intermediate step locations, which are achieved by energizing the coils with intermediate current levels \cite{step_book}.

The last major consideration in driving steppers is controlling the start-up and stopping speeds for the controllers provided pulse train to the driver. As the motor is a mechanical device in the real world, expecting a perfect impulse response will lead to driving failure. Inertia ratio is critical to stepper motor acceleration \cite{step_book}. Too great a difference in inertia ratio between system and motor risks missed steps or stalling the coils. So when starting a stepper motor, acceleration and deceleration should happen through pulses to the motor that start slowly and gradually quicken in a process called ramping.

\section{Revised Approach}
What's the plan to solve what problems with what methods:

Focus on system automating the experimental process (motorised droplet position, dispensing and refilling), minimising variables that affect the repeatability of the experiment: Ie drop volume (pipette), drop position relative to substrate centre (programmed motor).