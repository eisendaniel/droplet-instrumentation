%% $RCSfile: proj_report_outline.tex,v $
%% $Revision: 1.3 $
%% $Date: 2016/06/10 03:41:54 $
%% $Author: kevin $

\documentclass[11pt, a4paper, oneside]{report}

\usepackage{float} % lets you have non-floating floats
\usepackage{url} % for typesetting urls
\usepackage{graphicx}
\usepackage{subcaption}
\usepackage{hyperref}
\usepackage{titlesec}
\titleformat{\chapter}[hang] 
{\normalfont\huge\bfseries}{\chaptertitlename\ \thechapter:}{1em}{} 
%
%  We don't want figures to float so we define
%
\newfloat{fig}{thp}{lof}[chapter]
\floatname{fig}{Figure}

%% These are standard LaTeX definitions for the document
%%                            
\title{Instrumentation System for Liquid Drop Impact and Evaporation}
\author{Daniel Eisen}

%% This file can be used for creating a wide range of reports
%%  across various Schools
%% 
%% Set up some things, mostly for the front page, for your specific document
%
% Current options are:
% [ecs|msor|sms]          Which school you are in.
%                         (msor option retained for reproducing old data)
% [bschonscomp|mcompsci]  Which degree you are doing
%                          You can also specify any other degree by name
%                          (see below)
% [font|image]            Use a font or an image for the VUW logo
%                          The font option will only work on ECS systems
% 
\usepackage[image,ecs]{vuwproject}

% You should specifiy your supervisor here with
\supervisor{Gideon Gouws}
% use \supervisors if there is more than one supervisor
 
\otherdegree{Bachelor of Engineering with Honours}

% Unless you've used the bschonscomp or mcompsci
%  options above use
%   \otherdegree{OTHER DEGREE OR DIPLOMA NAME}
% here to specify degree

% Comment this out if you want the date printed.
\date{}

\begin{document}

% Make the page numbering roman, until after the contents, etc.
\frontmatter
%%%%%%%%%%%%%%%%%%%%%%%%%%%%%%%%%%%%%%%%%%%%%%%%%%%%%%%
%%%%%%%%%%%%%%%%%%%%%%%%%%%%%%%%%%%%%%%%%%%%%%%%%%%%%%%

\begin{abstract}

    Droplet impact and evaporation presents a complex physical process worth investigating not only from a fundamental research perspective but also in its potential for industrial application. However, in order to extract usable data from this small scale, fast phenomenon in the lab producing a droplet of repeatable volume and position as well as accucartly track and collect the data (temperature, impact, evaporation) is essential in order to extract reliable and study worthy results. This procedure form the basis of this project.
    This projects approach is to add motorised automation to droplet dispensing with aims to control for droplet volume, positional variation, contact angle and speed up the procedure and allow for greater flexibility in the experimental process.


\end{abstract}

%%%%%%%%%%%%%%%%%%%%%%%%%%%%%%%%%%%%%%%%%%%%%%%%%%%%%%%

\maketitle

\include{acknowledge}

\tableofcontents


\mainmatter
%%%%%%%%%%%%%%%%%%%%%%%%%%%%%%%%%%%%%%%%%%%%%%%%%%%%%%%
%%%%%%%%%%%%%%%%%%%%%%%%%%%%%%%%%%%%%%%%%%%%%%%%%%%%%%%

% individual chapters included heretala
\chapter{Introduction}\label{C:intro}
The investigation of droplet impact and evaporation is an area of experimentation of interest and application to various industry. Such as milk powder spray drying, ink jet printing, and applications of evaporative cooling. This project will continue on from a previous instrumentation setup, evaluate its shortcomings, and design the next generation to improve the reliability and usability of the collected data and introduce methods of automating the process. 

Report covers the key problem the experimental setup that the project aims to improve, and what are the main factors that can introduce variation into the results. An overview and evaluation of the current/previous system and the variability of its results and possible sources of that inconsistency. It was briefly cover other similar systems in literature can what they control/how. Then it will redefine the project scope, its objectives and what variables its aim to control and how. Finally it will cover what work has been down and where the current progress and results stand as well as detailing difficulties/decisions made along then way.  

\section*{The Experiment}

A droplet of liquid (concentrated milk, water, etc) is deposited on a heated substrate (stainless steel, copper, glass). The temperature of the substrate in monitored, along with the progression of the droplets impact and evaporation being captured with 2 cameras; above and profile. This produces a multidimensional perception of the developing behaviour and characteristics of the droplet over time. This usually take 1-2mins per droplet.

\subsection*{Problems}

Such a process
\chapter{Background}\label{C:back}

\section{Instrumentation and Current System}

\section{Other Approaches}
In literature, there exists a variety of rigs for similar experiments. These were explored to gauge a range of what factors were controlled (and measured) and with what approaches. In summary:
\begin{itemize}
    \item Controlled environmental factors with basic box \cite{step_book} or measured factors only \cite{measure_only}.
    \item Controlled for droplet position and volume with hard mounted pump plus needle, but lacks environmental control or automation, and there is no top view camera \cite{non_newt} \cite{fixed_pump}.
    \item Incorporated stepper motor driven automation, but no environmental control or monitoring \cite{motors}.
    \item Full environmental chamber and fixed droplet pump, but single camera, has separate uncontrolled rig for top view \cite{duel_rig}.
\end{itemize}

\newpage
\section{Background on Stepper motor control}
This section will cover the background of controlling bipolar stepper motors via a step/direction style driver setup, as its concepts will be mentioned later in the report. This is a focused background on the key considerations and requirements when designing for and operating this specific subset, and by no means applicable to all driving and specific motor choices.

The stepper motors provide precise positioning and are capable of moving their rotor to a specified position and holding that position at a wide range of load torques. This capability makes the stepper motors popular in optics, medical instruments, factory automation, and industrial equipment.

The typical topology of a stepper driving system (based around the step/direction method) consists of a controller, driver, and stepper motor. The controller provides a direction signal and step pulses, while the driver converts these signals into actual electrical power and supplies them to the motor. The stepper motor moves in steps, each step covering one step angle, which can be described as the rotor displacement corresponding to one step pulse \cite{step_app}.

Stepper motors typically have a step size specification (e.g. 1.8° or 200 steps per revolution), which applies to full steps. Step/direction drivers usually provide a 'microstepping' mode which increases the resolution by allowing intermediate step locations, which are achieved by energising the coils with intermediate current levels \cite{step_book}.

The last major consideration in driving steppers is controlling the start-up and stopping speeds for the controller's provided pulse train to the driver. As the motor is a mechanical device in the real world, expecting a perfect impulse response will lead to driving failure. Inertia ratio is critical to stepper motor acceleration \cite{step_book}. Too great a difference in inertia ratio between system and motor risks missed steps or stalling the coils. So when starting a stepper motor, acceleration and deceleration should happen through pulses to the motor that start slowly and gradually quicken in a process called ramping.


\chapter{Work Done}\label{C:work}

\textit{outline the current work done, the designs, what's built, what ordered, why these design decision were made etc.}

\begin{itemize}
    \item 
    \item 
    \item 
    \item 
\end{itemize}
\chapter{Future Plan}\label{C:fut} 

\textit{This could highlight the main components which remain to be done, and provide a proposed time-line in which this will happen. In putting together a time line, students must take into account upcoming examinations, coursework deadlines and other disruptions. }
\chapter*{Feedback}\label{C:feed} 

\textit{This could highlight any difficulties currently faced, and make specific requests for guidance from the examination committee. For example, a student may be unsure how best to evaluate their artifact, and would appreciate suggestions for alternative methods. 
}

\backmatter
%%%%%%%%%%%%%%%%%%%%%%%%%%%%%%%%%%%%%%%%%%%%%%%%%%%%%%%
%%%%%%%%%%%%%%%%%%%%%%%%%%%%%%%%%%%%%%%%%%%%%%%%%%%%%%%

\bibliographystyle{ieeetran}
\bibliography{sample}

\end{document}
