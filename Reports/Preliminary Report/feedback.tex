\chapter*{Feedback}\label{C:feed} 

Evaluating the systems performance is now going to be two-fold. The first is as before, its performance as a new experimental rig. Does it produce more reliability and repeatability in its results, does its successfully control for the factors its has aim to control and this can be easily evaluated by comparison to the previous data collected on the old/current rig. But now I am aware the another goal of this project is the introduction of a level of automation that aim to increase the usability/flexibility of the instrumentation. What I am wondering is what would be an approach in evaluating the success of this aspect? Just that its successfully carries out a set of preprogramed motions and behaves as expects, the experiment is done etc?